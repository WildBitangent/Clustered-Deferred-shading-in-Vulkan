\chapter{Metódy vykreslovania veľa svetiel}
V tejto kapitole sú popísané metódy pre vykreslovanie mnoho svetieľ v reálnom čase. Obyčajná metóda, ako napr. forward shading v ktorej objekt postupne prechádza celou grafickou pipeline, je absolútne nedostačujúca pre potreby vykreslovania scény s mnoho svetlami v reálnom čase. Pre tento účel sa používa deferred shading \cite{def-vs-for}.

\section{Deferred shading}
\label{deferred shading}
Deferred shading (možný preklad ako odložené tieňovanie) je metóda, ktorá redukuje počet objektov, pre ktoré sa musí počítať osvetlovací model. Všetky objekty prejdú prvým priechodom, v ktorom sa vykreslia za použitia viacnásobných vykreslovacích cielov do jednotlivých geometrických bufferov, alebo v skratke G-bufferov. Do G-bufferov sa ukladajú jak farby daných fragmentov, tak normály, spekulárna zložka svetla, pozícia fragmentu (v mojom prípade transformovaná do view-space) a hĺbková mapa. Týmto spôsobom sa redukuje počet fragmentov, pre ktorý sa musí počítať osvetlenie na O(\textit{rozlíšenie~*~počet svetiel}). Samotný výpočet osvetlenia sa vykonáva ako 2D post proces. 

Nevýhodou tejto metódy je náročnosť na veľkosť grafickej pamäte (rastúcou s rozlíšením) a teda aj s priepustnosťou frame bufferu. Ďaľšou nevýhodou je nepodporovaný alpha blending, ktorý ale môže byť vyriešený použitím forward shading pre transparentné objekty~\ref{hybrid}.

\begin{figure*}[h]\centering
  \centering
  \includegraphics[width=0.33\linewidth]{obrazky-figures/placeholder.pdf}\hfill
  \includegraphics[width=0.33\linewidth]{obrazky-figures/placeholder.pdf}\hfill
  \includegraphics[width=0.33\linewidth]{obrazky-figures/placeholder.pdf}\hfill\\
  \includegraphics[width=0.33\linewidth]{obrazky-figures/placeholder.pdf}\hfill
  \includegraphics[width=0.33\linewidth]{obrazky-figures/placeholder.pdf}\hfill
  \includegraphics[width=0.33\linewidth]{obrazky-figures/placeholder.pdf}\hfill
  \caption{Obsah G-bufferov postupne z ľava: spojené g-buffere, albedo, normály, spekulárna zložka svetla, pozícia, hĺbková mapa.}
  \label{G-buffers}
\end{figure*}

\section{Tiled deferred shading}
Tiled shading je technika na redukovanie množstva svetelných výpočtov pre fragment. Jej princíp spočíva v nasledujúcich krokoch \cite{tiled-shading}:
\begin{itemize}
	\item Vykreslenie geometrie do G-bufferov
	\item Vytvorenie screenspace dlaždíc 
	\item Výpočítanie frustum a zistenie kolízií
	\item Vypočítanie osvetlenia
\end{itemize}

\subsection*{Vykreslenie geometrie do G-bufferov}
Jedná sa o klasický deferred rendering, kedy sa pomocou viacnásobných vykreslovacích cieľov vykreslí geometria do g-bufferov.

\subsection*{Vytvorenie screenspace dlaždíc}
Screen space sa rozsegmentuje na dlaždice s pevnou veľkosťou (typicky $32 \times 32$, alebo $16 \times 16$ pixelov), z ktorej každá obsahuje zoznam svetieľ, ktoré do nej patria a bude sa s nimi počítať osvetlovací model, pre fragmenty spadajúce do tejto dlaždice.

\subsection*{Výpočítanie frustum a zistenie kolízií}
Po vypočítaní frustum pre danú dlaždicu zisti svetlá, ktoré s ním kolidujú a pridaj ich do zoznamu kolidujúcich svetiel s danou dlaždicou.

\subsection*{Vypočítanie osvetlenia}
Pre každý fragment vo frame buffery zisti do akej dlaždice patrí a vypočítaj osvetlenie zo svetiel, ktoré boli pridané do zoznamu v minulom kroku.

\section{Clustered deferred shading}
\blind{10}

\section{Hybridný algoritmus}
\label{hybrid}
\todo{A mozno aj nie}